\documentclass[hyperref={pdfpagelabels=false}]{beamer}
\usepackage{lmodern}
\usepackage{url}
\usetheme{CambridgeUS}
\usepackage{amsmath}
\usepackage{amsthm}
\usepackage{enumerate}
\usepackage{relsize}
\usepackage{wasysym}
\usepackage{amssymb}

\theoremstyle{remark}
\newtheorem*{remark}{Remark}


\title{BaRK OPRF and Fuzzy PSI}
\author[Abhishek Kumar]{Abhishek Kumar\\{\small Supervised by: Dr. Bhavana Kanukurthi}}
\date{\today} 
\begin{document}
\logo{\includegraphics[scale=0.14]{iit} \ \ \includegraphics[scale=0.14]{iisc}}
\begin{frame}
\titlepage
\end{frame} 


\begin{frame}
\frametitle{Table of contents}
\tableofcontents
\end{frame} 


\section{Introduction} 
\begin{frame}
\frametitle{Introduction} 
\begin{itemize}
\item In this talk we will talk about a protocol for oblivious evaluation of psuedorandom function(OPRF). \pause
\item This mentioned protocol is efficient for evaluating a large batch of OPRF instances. \pause
\item We will consider semi-honest setting.\pause
\item The above protocol is due to \cite{KKRT} called Batched Related Key OPRF(BaRK-OPRF).\pause
\item PSI uses Oblivious Transfers' Extension(OT-Extension) from \cite{KK} as building blocks.
\end{itemize}
\end{frame}

\begin{frame}
\frametitle{Motivation}
\begin{itemize}
\item This work significantly improves the state-of-the-art Private Set Intersecction(PSI) protocol of \cite{PSSZ}.\pause
\item Among the problems of secure computation, PSI is most strongly motivated by practice. \pause
\item It is used for measuring Ad Conversion rate, Security incident information sharing, private contact discovery.\pause
\item The \cite{IKNP} OT extension is one of the gems in the field of secure computation and used in implementations of Yao and GMW protocols.
\end{itemize}
 
\end{frame}

\begin{frame}
\frametitle{Related Work}
\begin{itemize}
\item OPRFs were introduced by \cite{FIPR} but they use expensive public key operations. Also they use number of OTs proportional to bit length of PRF input. \pause
\item Just like standard OPRF, BaRK-OPRF efficiently implies keyword search functionality of \cite{FIPR}. \pause
\item The protocol of \cite{CNS} constructs an OPRF from unique blind signature schemes.
\end{itemize}
\end{frame}


\section{Preliminaries}
\subsection{Definitions}

\begin{frame}
\frametitle{Notations}
\begin{itemize}
\item We denote vectors in bold for eg. $\mathbf{a}$.
\item $OT_l^m$ means $m$ instances of OTs of $l$-bit strings.
\item We denote Matrices in capital say $M$.
\item For matrix $M$ 
	\begin{itemize}
	\item $M_i$ means $i^th$ row of M
	\item $M^j$ means $j^th$ coloumn of M.
	\end{itemize}
\item For vector $\mathbf{a}$ and bit $b$, $\mathbf{a} \cdot b$ means $\mathbf{a}$ if $b=1$ else $\mathbf{0}$.
\item For vectors  $\mathbf{a}$ and bit $\mathbf{b}$, $\mathbf{a} \cdot \mathbf{b}$ means bitwise AND.
\end{itemize}

\end{frame}


\begin{frame}
\frametitle{Correlation Robustness}

\begin{definition}[1]
\textit{Let H be a hash function with input length n. Then H is a d-\textbf{Hamming correlation robust} if for any strings $z_1....z_m \in \{0,1\}^*$ and $a_1,....a_m,b_1,....b_m \in \{0,1\}^*$ with $\|b_i\|_H \geq d $, the following distribution, induced by random sampling of $s \longleftarrow \{0,1\}^n$, is psuedo-random :}
\begin{equation*}
H(z_1||a_1 \oplus [b_1 \cdot s]),.....,H(z_m||a_m \oplus [b_m \cdot s])
\end{equation*}
%$H(z_1||a_1 \oplus [b_1 \cdot s])..... H(z_m||a_m \oplus [b_m \cdot s])$`
``." denotes bitwise-AND
\end{definition}
\end{frame}

\begin{frame}
\frametitle{Psuedo-Random Codes}
\begin{definition}[2]
\textit{Let $\mathcal{C}$ be a family of functions. We say that $\mathcal{C}$ is a ($d,\epsilon$) $\mathbf{psuedorandom}$ $\mathbf{code(PRC)}$ if for all strings $x \neq x^\prime$,}
\begin{equation*}
\Pr_{C\leftarrow \mathcal{C}} \left[ \|C(x) \oplus C(x^\prime)\|_H < d \right] \leq 2^{-\epsilon}
\end{equation*}
\end{definition}\pause

\begin{lemma}[1]
\textit{Suppose $G : \{0,1\}^\kappa \times \{0,1\}^* \rightarrow \{0,1\}^n$ is a psuedorandom function. Define $\mathcal{C} = \{G(s,\cdot) \  | \ s \in \{0,1\}^\kappa\}$. Then $\mathcal{C}$ is a $(d,\epsilon)$-psuedorandom code where : }

\begin{equation*}
2^{-\epsilon} = 2^{-n}\sum_{i=0}^{d-1} {n \choose i} + \vartheta(\kappa)
\end{equation*}
\end{lemma}
\end{frame}

\begin{frame}
\frametitle{High Level Idea}
\begin{itemize}
\item We see 1-out-of-$\infty$ OT extension as OPRF. The intuition begind this is the fact that Sender has ability to  evaluate the function at any point but remains oblivious to Receiver's choice. Also Receiver can't evaluate function at any point as it doesn't have access to key.\pause

\item However in the construction as we shall see, more information is leaked than required. Hence we will meet the definition of \textsf{relaxed OPRF} as defined in \cite{FIPR}.

\end{itemize}
\end{frame}

\begin{frame}
\frametitle{relaxed PRF}
\begin{definition}[3]
\textit{F is said to be a \textsf{relaxed PRF} if there is another $\widetilde{F}$, such that F(key,r) can be efficiently computed given just $\widetilde{F}$(key, r).}
\end{definition}\pause

\begin{definition}[4]
\textit{Let F be a relaxed PRF with output length v, for which we can write the $key = (k^*,k)$. Then F has $\mathbf{m-related-key-PRF (m-RK-PRF) security}$ if the advantage of any PPT adversary in the following game is negligible :
}
\end{definition}
\end{frame}
\subsection{Security game for relaxed-OPRF}
\begin{frame}
\frametitle{Security Game}
\begin{enumerate}
\item The adversary chooses a large no. of strings $x_1,.....x_n$ which it will query to $\widetilde{F}$.\pause
\item Then it selects $m$ pairs $(p_1,y_1),.....(p_m,y_m)$. where $p_i \neq x_{j_i}$.\pause
\item Challenger chooses keys $k^*, k_1, k_2, ....k_n$ which can be used as key to $F$.\pause

\end{enumerate}

\begin{remark}
Pair $(K^*,K_i)$ will be used as $key$ to $F$ for input $x_i$ for $i \in \left[n\right] $.
We assume that $p_1,...p_m \in \left[n\right]$. This is because $p_i's$ are basically meant to determine the key $(k^*,k_{p_i})$ that will be used when  input to $F$ is $y_i's$ for $y \in \left[m\right]$.
\end{remark}

\end{frame}

\begin{frame}

\begin{remark}
$y_i \neq x_{p_i}$ for $i \in \left[m\right]$. If this is not true then adversary will win trivially, as it means that adversary is choosing is a string which has been queried before with has been queried before with same key hence it can easily distinguish from random.
\end{remark}
\end{frame}
\begin{frame}
\frametitle{Game continued...}
\begin{enumerate}
\setcounter{enumi}{3}
\item Challenger tosses a coin $b \leftarrow \{0,1\}$.\pause
\begin{enumerate}[(a)]
\item If $b=0$, challenger outputs $\{\widetilde{F}((k^*,k_i),x_i)\}_i$ for $i \in \left[n\right]$ and $\{F((k^*,k_{p_l}),y_l)\}_l$ for $l \in \left[m\right]$.\pause
\item If $b=1$, challenger outputs $\{\widetilde{F}((k^*,k_i),x_i)\}_i$ for $i \in \left[n\right]$ and $\{z_l\}_l$ for $l \in \left[m\right]$ where $z_l \in_{randomly} \{0,1\}^v$.\pause
\end{enumerate}
\item Adversary outputs a bit $b^\prime$. Adversary wins if $b^\prime = b$\pause
\end{enumerate}
\textit{The PRF is secure if $\Pr\left[b=b^\prime\right] \leq \frac{1}{2} + negligible$} \pause
\begin{remark}
Intuitively the $PRF$ is instantiated with $n$ related keys(sharing $k^*$ value ). The adversary learns the relaxed output on one chosen input for each key. Then any $m$ additional $PRF$ outputs are indistinguishable from random.
\end{remark}
\end{frame}

\section{Result}
\begin{lemma}[2]
Let $\mathcal{C}$ be a $(d,\epsilon+\log{m})-PRC$ let  $H$ be $d-hamming\ correlation\ robust$ hash function. Let us define following relaxed PRF for $C \in_{randomly} \mathcal{C}:$
\begin{align*}
\mathlarger{\mathlarger{F(}} \mathlarger{(}  (C,s),(q_j,j)\mathlarger{)}, r \mathlarger{\mathlarger{)}} = \mathlarger{H(} j||q_j \oplus [C(r)\cdot s] \mathlarger{)}
\\
\mathlarger{\mathlarger{\widetilde{F}(}} \mathlarger{(}  (C,s),(q_j,j)\mathlarger{)}, r \mathlarger{\mathlarger{)}} = \mathlarger{(} j, C, q_j \oplus [C(r)\cdot s] \mathlarger{)}
\end{align*}
Then $F$ has $\mathbf{m-RK-PRF}$ security .
\end{lemma}
\subsection{Protocol}

\begin{frame}
\frametitle{Ideal Functionality we want to achieve :}
\begin{itemize}
\item Thre are two parties, psuedorandom function $F$ and corresponding $\widetilde{F}$ a number $m$ of instances.\pause
\item On input ($r_1,r_2,......r_m$) from receiver :
	\begin{itemize}
	\item Choose random components for seeds to $PRF$ : $k^*, k_1, k_2, ......k_m$ and give it to sender.\pause
	\item Give $\mathlarger{\mathlarger{\widetilde{F}(}} \mathlarger{(}  k^*,k_1, \mathlarger{)}, r_1 \mathlarger{\mathlarger{)}},.....\mathlarger{\mathlarger{\widetilde{F}(}} \mathlarger{(}  k^*,k_m \mathlarger{)}, r_m \mathlarger{\mathlarger{)}}$ to receiver.\pause
\end{itemize}	 
\end{itemize} 
\end{frame}

\subsection{BaRK OPRF Protocol}

\begin{frame}
\frametitle{BaRK OPRF Protocol}
\textbf{Input of R :} m selection strings $\mathbf{r} = (\mathbf{r}_1,.....\mathbf{r}_m), r_i \in \{0,1\}^*, i \in [m]$.\par \pause
\textbf{Parameters :}
\begin{itemize}
\item A ($\kappa ,\epsilon$)-PRC family $\mathcal{C}$ with output length $k=g(\kappa)$.\pause
\item A $\kappa-$Hamming Correlation Robust $H:[m] \times \{0,1\}^k \rightarrow \{0,1\}^v$ \pause
\item An Ideal $OT_m^k$ primitive.\pause
\end{itemize}
\textbf{Protocol :}\pause
\begin{enumerate}
\item S chooses a random $C \leftarrow \mathcal{C}$ sends to R.\pause
\item S chooses $\mathbf{s} \leftarrow \{0,1\}^k$ at random. Let $s_i$ denote $i^th$ bit of $\mathbf{s}$.\pause
\item R forms $m \times k$ matrix $T_0, T_1$ in following way :
\end{enumerate}

\end{frame}

\begin{frame}
\frametitle{BaRK OPRF Protocol}
\begin{enumerate}
\setcounter{enumi}{3}
\item[] \begin{itemize}
\item For $j \in [m]$, choose $\mathbf{T}_{0,j} \leftarrow \{0,1\}^k$ randomly.\pause
\item Set $\mathbf{T}_{1,j} = \mathbf{T}_{0,j} \oplus C(\mathbf{r}_j)$\pause
\end{itemize}
\item S and R interact with $OT_m^k$ in the following way :\pause
\begin{itemize}
\item S acts as receiver with input $\{s_i\}, i \in [k]$\pause
\item R acts as sender with input $\{\mathbf{T}_0^i,\mathbf{T}_1^i\}, i \in [k]$\pause
\item From the output S receives S forms $m\times k$ matrix $Q$ such that 
\begin{align*}
\mathbf{Q}^i &= \mathbf{T}_{s_i}^i, \ for\  i \in [k] \\
\mathbf{Q}_j &= \mathbf{T}_{0,j} \oplus (C(\mathbf{r}_j) \cdot \mathbf{s}) \ for \ j \in [m]
\end{align*}
\end{itemize}\pause
\item For $for \ j \in [m]$ S outputs $PRF$ seed $((C,\mathbf{s}),(j,\mathbf{Q}_j))$.\pause
\item For $for \ j \in [m]$ R outputs relaxed-$PRF$ output $(C,j,\mathbf{T}_{0,j})$.\pause
\end{enumerate}
From relaxed-$PRF$'s output, R can calculate PRF's output.
\end{frame}


\begin{frame}
\begin{center}
\Huge Thank You \smiley{}
\end{center}
\end{frame}

\section{Fuzzy Private Set Intersection}
\begin{frame}
\frametitle{something}
\end{frame}

\begin{frame}
\frametitle{something}
\end{frame}


\section{Appendix}
\begin{frame}
\frametitle{Proof of Lemma 1}
\begin{proof}
\textit{If we define $\mathcal{C}$ from random functions G instead of psuedorandom we get
\begin{equation*}
\Pr_{C\leftarrow \mathcal{C}} \left[ \|C(x) \oplus C(x^\prime)\|_H < d \right] = 2^{-n}\sum_{i=0}^{d-1} {n \choose i}
\end{equation*}
In the case when G is psuedorandom function this probability must be equal to
\begin{equation*}
2^{-n}\sum_{i=0}^{d-1} {n \choose i} + \vartheta(\kappa)
\end{equation*}
where $\vartheta(\kappa)$ is negligible
else we can build a distinguisher $\mathcal{D}$ that distinguishes output of PRF from Random function by calculating hamming weights of inputs.}
\end{proof}

\end{frame}

\begin{frame}
\frametitle{The IKNP construction for extending OT with semi-honest receiver}
Given a small number of oblivious transfers can we implement a large number of oblivious transfers ? \par \pause
The \cite{IKNP} construction does that \smiley{}.\par \pause
This protocol reduces reduces $OT_l^m$ to $OT_m^k$ which is ultimately reduced to $OT_k^k$ ($m >> k$). Overhead is few symmetric key operations.
\end{frame}

\begin{frame}
\frametitle{The IKNP construction for extending OT with semi-honest receiver}
\begin{itemize}
\item \textbf{Input of $S$}: $m$ pairs ($\mathbf{x}_{j,0}, \mathbf{x}_{j,1}$) of $l$-bit strings, $1 \leq j \leq m$. \pause
\item \textbf{Input of $R$}: $m$ selection bits of $\mathbf{r}$ = ($r_1,.....r_m$). \pause
\item \textbf{Common Input}: a security parameter $k$.\pause
\item \textbf{Oracle}: a random oracle $H : \left[m\right] \times \{0,1\}^k \rightarrow \{0,1\}^l$. \pause
\item \textbf{Cryptographic Primitive}: An ideal $OT_m^k$ primitive.\pause
\end{itemize}
\begin{enumerate}
\item $S$ initializes a random vector $\mathbf{s} \in \{0,1\}^k$ and $R$ random $m \times k$ bit matrix $T$ and $U$ such that $\mathbf{T}^i \oplus \mathbf{U}^i = \mathbf{r}$ for $i \in [k]$. \pause
\item The parties invoke the $OT_m^k$ primitive, where $S$ acts as receiver with input $s_i$ and $R$ as sender with inputs ($\mathbf{T}^i, \mathbf{r} \oplus \mathbf{T}^i$), $1 \leq i \leq k$,
\end{enumerate}
\end{frame}

\begin{frame}
\frametitle{The IKNP construction for extending OT with semi-honest receiver}
\begin{enumerate}
\setcounter{enumi}{2}
\item Let $Q$ denote the $m \times k$ matrix of values received by $S$. \pause

\begin{remark}
\begin{align}
\mathbf{Q}^i = (s_i \cdot \mathbf{r}) \oplus \mathbf{T}^i \\
\mathbf{Q}_j = (r_j \cdot \mathbf{s}) \oplus \mathbf{T}_j
\end{align}
\end{remark}\pause
\item $S$ sends ($\mathbf{y}_{j,0}, \mathbf{y}_{j,1}$) where \pause
\begin{align*}
\mathbf{y}_{j,0} &= \mathbf{x}_{j,0} \oplus H(j,\mathbf{Q}_j) \\
\mathbf{y}_{j,1} &= \mathbf{x}_{j,1} \oplus H(j,\mathbf{Q}_j \oplus \mathbf{s})
\end{align*}

\end{enumerate}
\end{frame}

\begin{frame}
\frametitle{The IKNP construction for extending OT with semi-honest receiver}

\begin{enumerate}
\setcounter{enumi}{2}
\item For $1 \leq j \leq m$, $R$ outputs $\mathbf{z}_{j} = \mathbf{y}_{j,r_j} \oplus H(j, \mathbf{T}_{j})$\pause
\end{enumerate}
We can easily verify that $\mathbf{z}_{j} =\mathbf{x}_{j,r_j}$
\end{frame}


\begin{frame}
\frametitle{The KK13 construction}
\begin{itemize}
\item If we observe \cite{IKNP} construction each row(say $i^{th}$) of $T \oplus U$ is either all 0's or 1's depending on $r_i$. \pause
\item Kolesnikov and Kumersan saw them as repetition code and came up with a similar protocol for extension of $1-out-of-2^l\ OT$ \pause
\item Now instead of choice bit $r_i$ receiver has choice string $\mathbf{r_i}$ of length $l$.\pause And let $C$ be an error correcting code of codeword length $k$. \pause
\item The receiver will prepare matrices $T$ and $U$ such that $\mathbf{T}_j \oplus \mathbf{U}_j = C(\mathbf{r_j})$.
\item Equation 2 will now become :
\begin{align}
Q_j = [ C(\mathbf{r_j}) \cdot \mathbf{s} ] \oplus \mathbf{T}_j
\end{align}

\end{itemize}
\end{frame}

\begin{frame}
\frametitle{The KK13 construction}
\begin{itemize}
\item \textbf{Input of $S$}: $m$ tuples ($\mathbf{x_{j,0},... x_{j,n-1}}$) of $l$-bit strings, $1 \leq j \leq m$. \pause
\item \textbf{Input of $R$}: $m$ selection integers of $\mathbf{r}$ = ($\mathbf{r}_1,.....\mathbf{r}_m$) such that $0 \leq \mathbf{r}_j \leq n-1$. \pause
\item \textbf{Common Input}: a security parameter $k$ and Walsh Hadamard Codes $C_{WH}^k$.\pause
\item \textbf{Oracle}: a random oracle $H : \left[m\right] \times \{0,1\}^k \rightarrow \{0,1\}^l$ . \pause
\item \textbf{Cryptographic Primitive}: An ideal $OT_m^k$ primitive.\pause
\end{itemize}
\begin{enumerate}
\item $S$ initializes a random vector $\mathbf{s} \in \{0,1\}^k$ and $R$ random $m \times k$ bit matrix $T$ and $U$ such that $\mathbf{T}_j \oplus \mathbf{U}_j = C(\mathbf{r_j})$. \pause
\item The parties invoke the $OT_m^k$ primitive, where $S$ acts as receiver with input $s_i$ and $R$ as sender with inputs ($\mathbf{T}^i, \mathbf{U}^i$), $1 \leq i \leq k$,
\end{enumerate}
\end{frame}

\begin{frame}
\frametitle{The KK13 construction}
\begin{enumerate}
\setcounter{enumi}{2}
\item Let $Q$ denote the $m \times k$ matrix of values received by $S$. \pause

\begin{remark}
\begin{align*}
\mathbf{Q}_j = (C(\mathbf{r_j}) \cdot \mathbf{s}) \oplus \mathbf{T}_j
\end{align*}
\end{remark}\pause

\item $S$ sends ($\mathbf{y}_{j,0},.... \mathbf{y}_{j,n-1} \ for \ j \in [m] \ and \ r \in [n-1] $) where \pause
\begin{align*}
\mathbf{y}_{j,r} &= \mathbf{x}_{j,r} \oplus H(j|| \mathbf{Q}_j \oplus (\mathbf{C(r)} \cdot \mathbf{s}) \\
\end{align*}
\item For $1 \leq j \leq m$, $R$ outputs $\mathbf{z}_{j} = \mathbf{y}_{j,\mathbf{r}_j} \oplus H(j, \mathbf{T}_{j})$\pause
\item it is easy to verify that R can't learn any other value and is able to learn value corresponding to his choice integer only.
\end{enumerate}
\end{frame}

\begin{frame}
\frametitle{Proof of lemma 2}
\begin{proof}\renewcommand{\qedsymbol}{}
In the $m-RK-PRF$ game with the defined $PRF$, we can write adversary's(controlling R) view as : 

\begin{equation}
\mathlarger{(} C, \{\mathbf{T}_{0,j}\}_{j \in [m]}, \{H(p_i ||\mathbf{T}_{0,p_i} \oplus [(C(\mathbf{x}_{p_i} ) \oplus C(\mathbf{y}_{i})) \cdot \mathbf{s}]  )\}_{i \in [m]} \mathlarger{)}
\end{equation}\pause
Now, in the game adversary chooses $m$ strings $y_i$. So we have $m$ terms of the form $C(x_{p_i}) \oplus C(y_i)$ for $x_{p_i} \neq y_i$. \pause
By our choice of $C$ each of these have hamming weight atleast $d$ with probability atleast $1-\frac{2^{-\epsilon}}{m}$.\\\pause
Let $A_i$ be the event when $i^th$ such term has hamming weight less than $d$. Then :\pause
\begin{align*}
\Pr \left[ A_1^\complement \cap A_2^\complement..... \cap A_m^\complement \right] &= 1 - \Pr \left[ (A_1^\complement \cap A_2^\complement..... \cap A_m^\complement)^\complement \right] \\
&= 1 - \Pr \left[  A_1 \cup A_2..... \cup A_m \right]
\end{align*}


\end{proof}
\end{frame}

\begin{frame}
\begin{proof}[Proof Continued]
By union bound

\begin{align*}
 \geq 1 - m \frac{2^{-\epsilon}}{m} = 1 - 2^{-\epsilon}.
\end{align*}\pause
So, by above calculation all such terms have hamming weight atleast $d$ with overwhelming probability($1 - 2^{-\epsilon}$).\\\pause
Conditioning on this event apply $d-$ hamming correlation robust property of $H$ to conclude that $H$ outputs are indistinguishable from random. 
\end{proof}
\end{frame}


\begin{center}
\vspace*{0.5px}
\huge{References}
\end{center}\par
\bibliography{talk}
\bibliographystyle{alpha}



\end{document}

