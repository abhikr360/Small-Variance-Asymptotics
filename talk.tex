\documentclass[hyperref={pdfpagelabels=false}]{beamer}
\usepackage{lmodern}
\usepackage{url}
\usetheme{CambridgeUS}

\title{Title of the Talk}  
%\author{Sascha Frank}
%Dr. Bhavana Kanukurthi
\author[Abhishek Kumar]{Abhishek Kumar\\{\small Supervised by: Dr. Bhavana Kanukurthi}}
\date{\today} 
\begin{document}
\logo{\includegraphics[scale=0.14]{iit} \ \ \includegraphics[scale=0.14]{iisc}}
\begin{frame}
\titlepage
\end{frame} 


\begin{frame}
\frametitle{Table of contents}
\tableofcontents
\end{frame} 


\section{Introduction} 
\begin{frame}
\frametitle{Introduction} 
\begin{itemize}
\item We will talk about a protocol for oblivious evaluation of psuedorandom function(OPRF). \pause
\item The above mentioned protocol is efficient for evaluating a large batch of OPRF instances. \pause
\item We will consider semi-honest setting.\pause
\item The above protocol is due to \cite{KKRT} called Batched Related Key OPRF(BaRK-OPRF).\pause
\item PSI uses Oblivious Transfers' Extension(OT-Extension) from \cite{KK} as building blocks.
\end{itemize}
\end{frame}

\begin{frame}
\frametitle{Motivation}
\begin{itemize}
\item This work significantly improves the state-of-the-art Private Set Intersecction(PSI) protocol of \cite{PSSZ}.\pause
\item Among the problems of secure computation, PSI is most strongly motivated by practice. \pause
\item It is used for measuring Ad Conversion rate, Security incident information sharing, private contact discovery.\pause
\item The \cite{IKNP} OT extension is one of the gems in the field of secure computation and used in implementations of Yao and GMW protocols.
\end{itemize}
 
\end{frame}

\begin{frame}
\frametitle{Related Work}
\begin{itemize}
\item OPRFs were introduced by \cite{FIPR} but they use expensive public key operations. Also they use number of OTs proportional to bit length of PRF input. \pause
\item Just like standard OPRF, BaRK-OPRF efficiently implies keyword search functionality of \cite{FIPR}. \pause
\item The protocol of \cite{CNS} constructs an OPRF from unique blind signature schemes.
\end{itemize}
\end{frame}


\section{Section no. 2} 
\subsection{Lists I}
\begin{frame}
\frametitle{unnumbered lists}
\begin{itemize}
\item Introduction to  \LaTeX{}  
\item Course 2 
\item Termpapers and presentations with \LaTeX{}  
\item Beamer class
\end{itemize} 
\end{frame}

\begin{frame}\frametitle{lists with single pauses}
\begin{itemize}
\item Introduction to  \LaTeX{}  \pause 
\item Course 2 \pause 
\item Termpapers and presentations with \LaTeX{}  \pause 
\item Beamer class
\end{itemize} 
\end{frame}

\begin{frame}\frametitle{lists with pause}
\begin{itemize}[<+->]
\item Introduction to  \LaTeX{}  
\item Course 2
\item Termpapers and presentations with \LaTeX{}  
\item Beamer class
\end{itemize} 
\end{frame}



\subsection{Lists II}
\begin{frame}\frametitle{numbered lists}
\begin{enumerate}
\item Introduction to  \LaTeX{}   
\item Course 2 
\item Termpapers and presentations with \LaTeX{}  
\item Beamer class
\end{enumerate}
\end{frame}

\begin{frame}
\frametitle{numbered lists with single pauses}
\begin{enumerate}
\item Introduction to  \LaTeX{}  \pause 
\item Course 2 \pause 
\item Termpapers and presentations with \LaTeX{}  \pause 
\item Beamer class
\end{enumerate}
\end{frame}

\begin{frame}
\frametitle{numbered lists with pause}
\begin{enumerate}[<+->]
\item Introduction to  \LaTeX{}  
\item Course 2
\item Termpapers and presentations with \LaTeX{}  
\item Beamer class
\end{enumerate}
\end{frame}




\section{Section no.3} 
\subsection{Tables}
\begin{frame}
\frametitle{Tables}
\begin{tabular}{|c|c|c|}
\hline
\textbf{Date} & \textbf{Instructor} & \textbf{Title} \\
\hline
WS 04/05 & Sascha Frank & First steps with  \LaTeX  \\
\hline
SS 05 & Sascha Frank & \LaTeX \ Course serial \\
\hline
\end{tabular}
\end{frame}


\begin{frame}
\frametitle{Tables with pause}
\begin{tabular}{c c c}
A & B & C \\ 
\pause 
1 & 2 & 3 \\  
\pause 
A & B & C \\ 
\end{tabular} 
\end{frame}


\section{Section no. 4}
\subsection{blocs}
\begin{frame}
\frametitle{blocs}

\begin{block}{title of the bloc}
bloc text
\end{block}

\begin{exampleblock}{title of the bloc}
bloc text
\end{exampleblock}


\begin{alertblock}{title of the bloc}
bloc text
\end{alertblock}
\end{frame}

\section{References}
\bibliography{talk}
\bibliographystyle{alpha}



\end{document}

